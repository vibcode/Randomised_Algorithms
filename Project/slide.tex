\documentclass{beamer}
%
% Choose how your presentation looks.
%
% For more themes, color themes and font themes, see:
% http://deic.uab.es/~iblanes/beamer_gallery/index_by_theme.html
%
\mode<presentation>
{
  \usetheme{Madrid}      % or try Darmstadt, Madrid, Warsaw, ...
  \usecolortheme{beaver} % or try albatross, beaver, crane, ...
  \usefonttheme{serif}  % or try serif, structurebold, ...
  \setbeamertemplate{navigation symbols}{}
  \setbeamertemplate{caption}[numbered]
} 

\usepackage[english]{babel}
\usepackage[utf8x]{inputenc}
\usepackage{xcolor}
\usepackage{listings}
\usepackage{algorithm2e}
\lstset
{
    language=[LaTeX]TeX,
    breaklines=true,
    basicstyle=\tt\scriptsize,
    %commentstyle=\color{green}
    keywordstyle=\color{blue},
    %stringstyle=\color{black}
    identifierstyle=\color{magenta},
}

\title[CS648]{Smallest Enclosing Circle Problem}
\author{Raghukul,Vibhor}
\institute{IIT Kanpur}
\date{\today}

\AtBeginSection[]
{
  \begin{frame}<beamer>
    \frametitle{Outline}
    \tableofcontents[currentsection,currentsubsection]
  \end{frame}
}

\begin{document}

\begin{frame}
  \titlepage
\end{frame}

% Uncomment these lines for an automatically generated outline.
\begin{frame}{Outline}
  \tableofcontents
\end{frame}

\section{Introduction}

\begin{frame}{Problem}
  \begin{itemize}
      \item Given $n$ points in a plane, compute the smallest radius circle enclosing all $n$ points.
      \pause
      \item The time complexity of best deterministic algorithm for this problem is $O(n)$ which uses advanced geometry.
      \pause 
      \item We will present a simple randomised algorithm for this problem with expected $O(n)$ time complexity. 
      \pause
    \end{itemize}
\end{frame}

\begin{frame}{Ideas}
  \begin{itemize}
      \item A circle is completely determined if we are given three or two points lying on its boundary.
      \pause
      \item We can use this idea to get a trivial $O(n^4)$ deterministic algorithm for our problem.
      \pause
      \item This algorithm can be easily modified to give a deterministic algorithm with $O(n^3)$ time complexity.
      \pause
      \item The idea is that given two defining points we can find the third defining point in $O(n)$ time.
      
      \pause
      \item Check whether these two circles(one constructed taking the 2 defining points as diameteric ends) are valid and return the smallest radius circle enclosing these points.
  \end{itemize}
\end{frame}

\section{Algorithm and Proofs}

\begin{frame}{$O(n^4)$ algorithm}
\scalebox{0.8}{
    \begin{minipage}{1\linewidth}
\begin{algorithm}[H]
\SetAlgoLined
\KwIn{$N$ points in $2D$ plane}
\KwOut{Circle enclosing given $N$ points and having minimum radius}
 $P \leftarrow$  list of $N$ points\\
 $C \leftarrow$ circle formed by taking $P[1], P[2]$ as diameter. \\
 \For{$i \leftarrow 1$ \KwTo $n$} {
  \For{$j \leftarrow i+1$ \KwTo $n$} {
    $C^{'} \leftarrow$ circle formed by taking $P[i], P[j]$ as diametric ends. \\
    \uIf{$C^{'}.rad <$  $C.rad$}{
      $C = C^{'}$
    }
  }
 }
 \For{$i \leftarrow 1$ \KwTo $n$} {
  \For{$j \leftarrow i+1$ \KwTo $n$} {
    \For{$k \leftarrow j+1$ \KwTo $n$} {
      $C^{'} \leftarrow$ Circumcircle of $P[i], P[j], P[k]$ \\
        \uIf{$C^{'}.rad <$  $C.rad$}{
          $C = C^{'}$
        }
    }
  }
 }
 \Return{$C$}
\end{algorithm}
\end{minipage}%
    }
\end{frame}



\begin{frame}{$O(n^3)$ algorithm}
\begin{algorithm}[H]
\SetAlgoLined
\KwIn{$N$ points in $2D$ plane}
\KwOut{Circle enclosing given $N$ points and having minimum radius}
 $P \leftarrow$  list of $N$ points\\
 $C \leftarrow$ circle formed by taking $P[1], P[2]$ as diameter. \\
 \For{$i \leftarrow 1$ \KwTo $n$} {
  \For{$j \leftarrow i+1$ \KwTo $n$} {
    $P_{min}$ = point subtending least acute angle on line $P[i]\rightarrow P[j]$\\
    $C_{1} \leftarrow$ circle formed by taking $P[i], P[j]$ as diametric ends. \\
    $C_{2} \leftarrow$ circumcircle of $P[i], P[j], P_{min}$. \\
    $C^{'} \leftarrow$ circle among $C_1, C_2$ having least radius. \\
    \uIf{$C^{'}.rad <$  $C.rad$}{
      $C = C^{'}$
    }
  }
 }
 \Return{$C$}
\end{algorithm}
\end{frame}


\begin{frame}{Expected $O(n)$ algorithm}
\begin{algorithm}[H]
\SetAlgoLined
\KwIn{$N$ points in $2D$ plane}
\KwOut{Circle enclosing given $N$ points and having minimum radius}
 $P \leftarrow$  list of $N$ points\\
 $P^{'} \leftarrow$ random permutation of $P$ \\ 
 $C \leftarrow$ circle formed by taking $P^{'}[1], P^{'}[2]$ as diameter. \\
 \For{$i \leftarrow 3$ \KwTo $n$} {
    \uIf{$C.outside(P^{'}[i])$}{
      $C \leftarrow$ build\_circle($i-1, P[i]$)
    }
 }

 \Return{$C$}
\end{algorithm}
\end{frame}


\begin{frame}{build\_circle}
\begin{algorithm}[H]
\SetAlgoLined
\KwIn{$k$, $P_d$}
\KwOut{Circle enclosing first $k$ points and $P_d$ being on boundary}
  $C \leftarrow$ circle formed by taking $P^{'}[1], P_d$ as diameter. \\
  \For{$i \leftarrow 2$ \KwTo $k$} {
    \uIf{$C.outside(P^{'}[i])$}{
        $l \leftarrow$ line joining $P^{'}[i], P_d$ \\
        $P_{min}$ = point($[1,i-1]$) subtending least acute angle on $l$\\
        $C_{1} \leftarrow$ circle formed by taking $P^{'}[i], P_d$ as diametric ends. \\
        $C_{2} \leftarrow$ circumcircle of $P^{'}[i], P_d, P_{min}$. \\
        $C \leftarrow$ circle among $C_1, C_2$ having least radius. \\
    }
 }
 \Return{$C$}
\end{algorithm}
\end{frame}

\begin{frame}{Lemma 1}
If a point $P$ is outside the smallest enclosing circle of set $S$ then it must be one of the defining points of smallest enclosing circle of set $S\cup \{P\}$.
\pause
\\
\textbf{Proof : } Note that there has to to at least 2 point on the circle. If not then we can compress the circle, so that there are at least 2 points on the circle.\\
\pause
Also the boundary points define the circle completely, if not, then again we can compress the circle. Suppose $P$ is not a defining point, by above statement there are some defining point on this new circle. These defining points were present before $P$ was added, so the radius of circle was at least equal to the radius of circle defined by these points. After adding $P$ they are defining point, so the radius of this equal to the radius defined by these point. In other words the circle didn't change, after adding $P$. But this contradicts our assumption that $P$ is outside the smallest enclosing circle.
% \pause
\end{frame}

\begin{frame}{Lemma 2}
Probability that point $P_i$ lies outside the smallest enclosing circle of points $P_1, ... , P_{i-1}$ is $\leq \frac{3}{i}$.
\pause
\\
\textbf{Proof : } By Lemma 1, if the smallest enclosing circle is changed on adding the $i^{th}$ point then it must be one of the defining points of smallest enclosing circle of first $i$ points, call this circle $C_i$.
\\
\pause
There can be at most $3$ defining points for $C_i$ because $C_{i-1}$ is not same as $C_i$, so the proability that $i^{th}$ point is defining point for $C_i$ is $\leq \frac{3}{i}$.

\end{frame}
\begin{frame}{Lemma 3}
 Given two defining points then the third defining point is the point which form the least acute angle with these two points.
\pause
\\
\textbf{Proof : } The angle subtended by a chord on the boundary of the circle is least compared to all other points in the same sector and for a point to be the defining point it must lie on the boundary.
\\
\pause
So the third defining point(if any) is the point which form the least acute angle with these two points.    
\end{frame}

\section{Analysis and Experimental results}

\begin{frame}{Analysis}
\begin{itemize}
    \item Worst case time complexity of our randomized algorithm is $O(n^3)$.\\ 
    \pause 
    This occurs when all the points lie outside the existing circle, and while updating circle, removing $q$ does not give a defining point on every step of recursion.
    \pause
    \item Expected time complexity of our randomized algorithm is $O(n)$.
    \pause
    \begin{itemize}
        \item The expected runtime of $build\_circle()$ function is $O(i)$.
    \end{itemize}
\end{itemize}
\end{frame}

\begin{frame}{Experimental Results}
\textbf{Comparison between run time of the $3$ algorithms discussed.}
\begin{table}
\centering
\begin{tabular}{|c| c |c |c|} 
 \hline\hline
 $N$ & Expected $O(n)$ algorithm & $O(n^3)$ Algorithm & $O(n^4)$ Algorithm \\ [0.5ex] 
 \hline\hline
10 & 0.063 & 0.186 & 0.248 \\ \hline
20 & 0.088 & 1.112 & 3.203 \\ \hline
30 & 0.101 & 3.395 & 14.635 \\ \hline
40 & 0.132 & 7.888 & 44.322 \\ \hline
50 & 0.153 & 15.107 & 105.807 \\ \hline
60 & 0.190 & 26.222 & 219.731 \\ \hline
70 & 0.206 & 42.041 & 408.725 \\ \hline
80 & 0.223 & 62.511 & 695.651\\ \hline
90 & 0.229 & 88.957 & 1109.614 \\ \hline
100 & 0.261 & 122.576 & 1700.718\\ \hline
\end{tabular}
\caption{All the values given above are in ms. Tested over $50$ randomly generated examples for each $N$.}
\label{Talbe}
\end{table}
\end{frame}


\begin{frame}{Experimental Results}
\textbf{Comparison between $O(n^3)$, $O(n)$ algorithms.}
\begin{table}
\centering
\begin{tabular}{|c| c |c |} 
 \hline\hline
 $N$ & Expected $O(n)$ algorithm & $O(n^3)$ Algorithm \\ [0.5ex] 
 \hline\hline
100 & 0.261 & 121.167 \\ \hline
200 & 0.437  & 959.231\\ \hline

300 & 0.631 & 3163.430
 \\ \hline

400 & 0.945 & 7615.576
 \\ \hline

500 & 1.040 & 14999.119
 \\ \hline

600 & 1.318 & 25769.805
 \\ \hline

700 & 1.703 & 43088.605
\\ \hline

800 & 1.712 & 66676.151
\\ \hline

900 & 1.876 & 93234.026
 \\ \hline

1000 & 2.188 & 120870.487
\\ \hline
\end{tabular}
\caption{All the values given above are in ms. Tested over $20$ randomly generated examples for each $N$.}
\end{table}
\end{frame}

\begin{frame}{Experimental Results}
\textbf{Statistics of $O(n)$ algorithm.}
\begin{table}
\centering
\begin{tabular}{|c| c |c |c|} 
 \hline\hline
 $N$ & Average Time & Worst Case Time & Standard deviation \\ [0.5ex] 
 \hline\hline
10 & 0.067  &  0.098  &  0.0114
 \\ \hline
100 & 0.277  &  0.737  &  0.127
 \\ \hline
1000 & 2.181  &  4.328  &  0.729
 \\ \hline
10000 & 19.894  &  48.375  &  8.953
 \\ \hline
100000 & 215.114  &  739.233  &  109.248
 \\ \hline
1000000 & 2259.028  &  4374.806  &  1121.353
 \\ \hline
\end{tabular}
\caption{All the values given above are in ms. Tested over $50$ randomly generated examples for each $N$.}\end{table}
\end{frame}

\begin{frame}{Experimental Results}
\textbf{Number of time new point lies outside current circle.}
\begin{table}
\centering
\begin{tabular}{|c| c|} 
 \hline\hline
 $N$ & no. of times oint lies outside (Avg value) \\ [0.5ex] 
 \hline\hline
10 & 3.667
 \\ \hline
100 & 8.529
 \\ \hline
1000 & 14.843
 \\ \hline
10000 & 21.235
 \\ \hline
100000 & 28.273
 \\ \hline
1000000 & 33.727
 \\ \hline
\end{tabular}
\caption{Tested over $50$ randomly generated examples for each $N$.}\end{table}
\end{frame}


\begin{frame}{Experimental Results}
\textbf{Percentage exceeding Average run time.}
\begin{table}
\centering
\resizebox{\textwidth}{!}{
\begin{tabular}{|c| c| c | c | c | c | } 
 \hline\hline
 Average time exceeded by(\%) & 10 & 100 & 1000 & 10000 & 100000 \\ [0.5ex] 
 \hline\hline
10 & 23.529 & 31.373  & 37.255 & 29.412 & 33.333
 \\ \hline
20 & 9.804& 25.49& 29.412 & 25.49 & 29.412 
 \\ \hline
50 & 0.0 & 9.804  &15.686 & 15.686 &  19.608
 \\ \hline
100 & 0 &  1.961 & 3.922&1.961 & 3.922
 \\ \hline
\end{tabular}
}
\caption{Tested over $250$ randomly generated examples for each $N$.}\end{table}
\end{frame}

\end{document}

